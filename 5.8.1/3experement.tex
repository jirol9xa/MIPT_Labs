
% Experement data + calcs goes here %

\section{Ход работы}

\subsection{Изучение работы оптического пирометра}
\begin{center}
    
\begin{tabular}{c|c|c}
     $T, K$ & $V, mV$ & $T_{table}, K$\\ \hline
     1191 & 45,93 &  1157\\ \hline
     1201 & 46,16 & 1163\\  \hline
     1207 & 46,19 & 1164\\ \hline
\end{tabular}
\end{center}

Проверили, что различия в значениях термодинамической и яркостной температуры АЧТ
менее $ 10\% $.

\subsection{Измерение яркостной температуры тел}

Убедились, что различные тела, нагретые до одинаковой термодинамической температуры
могут иметь различную яркостную температуру - использовали нагретые кольца с различными
к-тами отражения. Данную проверку получилось провести только посредством визуального
наблюдения - яркостная температура некоторых колец была слишком мала для измерения с
помощью пирометра. Фотография колец предоставлена на рисунке 2. 

\begin{center}
$T = 824 K$    
\end{center}

\pic{0.8\linewidth}{rings.jpg}{Наблюдаемые колца}

\subsection{Проверка закона Стефана-Больцмана}

Постепенно меняя накал нити лампы в диапазоне $ 900-1800^\circ C $ измерили
пирометром яркостную температуру нити накала лампы, а так же значение силы тока
и напряжения на ней. Определили так же по значениям яркостной температуры нити ее
термодинамическую температуру, используя зависимость $ T \left( T\ruB{ярк} \right) $
(см. рисунок 1). \\

Полученные значения представлены в таблице 1.

\begin{table}[h!]
\begin{center}

    \tableLable{Данные}
    \begin{tabular}{|c|c|c|c|c|c|c|c|c|c|c|}
    \hline
    $ t, ^\circ C $          & 900   & 1000  & 1100  & 1200  & 1300  & 1400  & 1500  & 1600  & 1700  & 1800  \\ \hline
    $ t_{table} \, ^\circ C $& 920   & 1030  & 1140  & 1250  & 1350  & 1450  & 1550  & 1660  & 1770  & 1880  \\ \hline
    $ I, \, $ А              & 0,433& 0,471& 0,521& 0,566& 0,633& 0,714& 0,780& 0,844& 0,982& 1,098\\ \hline
    $ V, \, $ В              & 1,35& 1,65& 2,08& 2,49& 3,15& 4,01& 4,76& 5,54& 7,40& 9,15\\\hline
    \end{tabular}

\end{center}
\end{table}

$W(T) = V \cdot I$

Построил зависимость $ W \left( T \right) \, $ в логарифмическом масштабе. Зависимость
предоставлена на рисунке 3.

\pic{0.8\linewidth}
{график.png}
{Зависимость $ W \left( T \right) \, $ в логарифмическом масштабе}

Получил коэффициент наклона: \\

$ n = 4,07 \pm 0,17 $ \\

Посчитал постоянную Стефана-Больцмана для $ 1700 ^\circ C \, $ и $ 1800 ^\circ C \, $: \\

$ \sigma_{1700} = \left( 6,3 \pm 0,9 \right) \cdot 10^{-5} \,
\frac{\text{эрг}} {\text{с} \cdot \text{см}^2 \cdot \text{К}^4} $ \\

$ \sigma_{1800} = \left( 6,7 \pm 0,9 \right) \cdot 10^{-5} \,
\frac{\text{эрг}} {\text{с} \cdot \text{см}^2 \cdot \text{К}^4} $ \\

\subsection{Измерение яркостной температуры неоновой лампы}

Термодинамическая температура лампы приблизительно равна комнатной. Однако ее яркостная
температура составляет $ \approx 843^\circ C $. Данное расхождение температур обусловлено
тем, что неоновая лампа не удолетворяет модели АЧТ.

\section{Вывод}

В ходе данной работы был проверен закон Стефана-Больцмана и границы его применимости. Успех!
Точность измерений можно повысить, заменив пирометр на более удобный (электронный).
